\documentclass{jsarticle}
\usepackage{ascmac}
\usepackage{amsmath}
\usepackage{amsmath}
\usepackage{amssymb}
\usepackage{amsfonts}

\begin{document}

mod p上の1の$n$乗根 出典:https://twitter.com/kirika\_comp/status/1203603433455927297

1の$n$乗根$a$を求めたい。$a^n=1 \bmod p$かつ$a^{p-1}=1\bmod p$だから$p-1 = 0 \bmod n $が必要。
このとき原始根$g$を用いて$n$乗根が$g^{(p-1)/n}$と書ける。
よって$p-1=0\bmod n$が$1$の$n$乗根が存在するための必要十分条件。
$b$をランダムに取ってくる。$b$が原始根$g$を用いて$b=g^k$と書けるとする。
$(k\frac{p-1}{n}x=0 \bmod p-1\Leftrightarrow x=n \bmod p-1)$は$k\mod n$が$n$と互いに素であることと同値。
このような$k$は$1,2,\ldots,p-1$のうち$\frac{p-1}{n}\phi(n)$個ある。したがって$\frac{\phi(n)}{n}$の確率で$b^{(p-1)/n}$が$1$の$n$乗根になる。

体係数1変数多項式環$K[X]$
ユークリッド整域だから拡張ユーグリッドの互除法により互いに素な$f,g$に対して$f^{-1} \bmod g$が求められる。従ってGarnerのアルゴリズムが適用できる。


\section{形式的冪級数} 
出典:http://sugarknri.hatenablog.com/entry/2019/10/08/001359
Rを可換環とする。$n-1$次で打ち切った形式的冪級数$P=R[[X]]/\langle X^{n} \rangle$の成す環の演算を考える。


\subsection{等比級数による逆元の計算}

等比級数の和の公式より
$$1/f=(f_0)^{-1}\sum_{i=0}^n(1-f)^i$$
である。
$g:=(1-f), h(k):=1+g+g^2+...+g^{2^{k}-1}$と置くと、
$h(k)=h(k-1)(1+g^{2^{k-1}})$という漸化式が成り立ち$O(n \log^2(n))$で計算できる。

\subsection{Newton法による逆元の計算}

\subsection{Newton法による平方根の計算}

$g^2=f $なる$g$を求めたい。
$F(X)=X^2-f$に対して$F(X)=0$の解をニュートン法で求めると
$$g_{n+1}=\frac{g_n}{2}+\frac{f}{2g_n}$$
となる。このとき

\begin{align}
  g_{n+1}^2-f=&(\frac{g_n}{2}+\frac{f}{2g_n})^2-f\\
  =&\frac{1}{4g_n^2}(g_n^2-f)^2\\
\end{align}

だから二次収束する。計算量は$O(\sum_{k=1,2,4,8,\ldots,n}k\log k)=O(n\log n)$となる。

\subsection{Newton法による対数の計算}

$\exp(g)=f$なる$g$を求めたい。$[x^0]f=1$とする。


\begin{align}
  &\exp(g)=f\\
  \Rightarrow&g'f=f'\\
  \Rightarrow&g=\int \frac{f'}{f} dX'\\
\end{align}

ただし$[x^0]g=0$である。よって$g$は$O(n \log n)$で求まる。

\subsection{Newton法による指数の計算}

$f=\exp(g) $を求めたい。$[X^0]g=0$とする。
$F(X)=\log(X)-f$として$F(X)=0\bmod X^{n+1}$の解をニュートン法で求めると
$$g_{n+1}=g_n(1-F(g_n))$$
となる。$\log(1+X)=X-\frac{X^2}{2}+O(X^3)$を用いて、
\begin{align}
  F(g_{n+1})&=\log(g_{n+1})-f\\
  &=\log(g_{n})+\log(1-F(g_n))-f\\
  &=\log(g_{n})-F(g_n)+\frac{g_n^2}{2}-f+O(g_n^3)\\
  &=\frac{g_n^2}{2}+O(g_n^3)\\
\end{align}
よって二次収束する。

\subsection{初等関数による合成関数}

$1,2,\ldots,n$が逆元を持つとする。このとき積分が計算できる。
よって$\log,\arctan$に対する合成関数は
$\log(f)=\int \frac{f'}{f}$, $\arctan(f)=\int \frac{f'}{1+f^2}$
によって計算できる。$\sin,\cos,\sinh,\cosh$の合成関数は$\exp$の線形結合に変形することで計算できる。
$\sin,\cos$については虚数単位$\sqrt{-1}$が必要になるので$R=\mathbb{F}_p(\sqrt{-1})$で計算する。
ただし平方剰余の相互法則の第一補充法則より$p\in 4\mathbb{Z}+1$のとき$\sqrt{-1}\in\mathbb{F}_p$であることに注意する。


\subsection{一般的な合成関数: Brent-Kung algorithm}

参考:http://fredrikj.net/math/rev.pdf

ホーナー法により

\begin{align}
  f(g)&=\sum_{k=0} f_k(g)^k\\
      &=f_0+g(f_1+g(f_2+\ldots))\\
\end{align}
とできて$O(n^2\log n)$で計算できる。

平方分割により高速化できる。$m=[\sqrt{n}]$として$h_k:=\sum_{i=0}^{m-1} f_{mk+i}g^i$とすると
$f=\sum_{i=0}^m h_i g^{mi}$
とできる。$g^i$の列挙は$O(n^{3/2} \log{n})$で行える。
愚直にやっても$h_k$の列挙は$O(n^{2})$で行える。
よって全体で$O(n^2)$で計算できる。


\end{document}
