\documentclass{jsarticle}
\usepackage{ascmac}
\usepackage{amsmath}
\begin{document}

mod p上の1の$n$乗根 出典:https://twitter.com/kirika_comp/status/1203603433455927297

1の$n$乗根$a$を求めたい。$a^n=1 \bmod p$かつ$a^{p-1}=1\bmod p$だから$p-1 = 0 \bmod n $が必要。
このとき原始根$g$を用いて$n$乗根が$g^{(p-1)/n}$と書ける。
よって$p-1=0\bmod n$が$1$の$n$乗根が存在するための必要十分条件。
$b$をランダムに取ってくる。$b$が原始根$g$を用いて$b=g^k$と書けるとする。
$(k\frac{p-1}{n}x=0 \bmod p-1\Leftrightarrow x=n \bmod p-1)$は$k\mod n$が$n$と互いに素であることと同値。
このような$k$は$1,2,\ldots,p-1$のうち$\frac{p-1}{n}\phi(n)$個ある。したがって$\frac{\phi(n)}{n}$の確率で$b^{(p-1)/n}$が$1$の$n$乗根になる。

体係数1変数多項式環$K[X]$
ユークリッド整域だから拡張ユーグリッドの互除法により互いに素な$f,g$に対して$f^{-1} \bmod g$が求められる。従ってGarnerのアルゴリズムが適用できる。


形式的冪級数 出典:http://sugarknri.hatenablog.com/entry/2019/10/08/001359

$g^2=f \bmod X^{n+1}$なる$g$を求めたい。
$F(X)=X^2-f$に対して$F(X)=0$の解をニュートン法で求めると
$$g_{n+1}=\frac{g_n}{2}+\frac{f}{2g_n}$$
となる。このとき

\begin{align}
  g_{n+1}^2-f=&(\frac{g_n}{2}+\frac{f}{2g_n})^2-f\\
  =&\frac{1}{4g_n^2}(g_n^2-f)^2\\
\end{align}

だから二次収束する。計算量は$O(\sum_{k=1,2,4,8,\ldots,n}k\log k)=O(n\log n)$となる。


$\exp(g)=f \bmod X^{n+1}$なる$g$を求めたい。$[x^0]f=1$とする。


\begin{align}
  &\exp(g)=f\\
  \Rightarrow&g'f=f'\\
  \Leftrightarrow&g=\int \frac{f'}{f} dX'\\
\end{align}

ただし$[x^0]g=0$である。よって$g$は$O(n \log n)$で求まる。

$f=\exp(g) \bmod x^{n+1}$を求めたい。$[x^0]g=0$とする。
$F(X)=\log(X)-f$として$F(X)=0\bmod X^{n+1}$の解をニュートン法で求めると
$$g_{n+1}=g_n(1-F(g_n))$$
となる。$\log(1+X)=X-\frac{X^2}{2}+O(X^3)$を用いて、
\begin{align}
  F(g_{n+1})&=\log(g_{n+1})-f\\
  &=\log(g_{n})+\log(1-F(g_n))-f\\
  &=\log(g_{n})-F(g_n)+\frac{g_n^2}{2}-f+O(g_n^3)\\
  &=\frac{g_n^2}{2}+O(g_n^3)\\
\end{align}
よって二次収束する。


\end{document}
